% !TEX program = xelatex

\documentclass{resume}
%\usepackage{zh_CN-Adobefonts_external} % Simplified Chinese Support using external fonts (./fonts/zh_CN-Adobe/)
%\usepackage{zh_CN-Adobefonts_internal} % Simplified Chinese Support using system fonts

\begin{document}
\pagenumbering{gobble} % suppress displaying page number

\name{Jiajun "Thomas" Liang}

\basicInfo{
  \email{toml@usc.edu} \textperiodcentered\ 
  \phone{(+1) 213-713-7499} \textperiodcentered\ 
  \linkedin[Thomas Liang]{https://www.linkedin.com/in/thomas-liang-52270823b/} \textperiodcentered\
  \github[ThmsLiang]{https://github.com/ThmsLiang}}

\section{Education}
\datedsubsection{\textbf{University of Southern California}, Los Angeles, California}{Aug 2020 -- May 2024}
\textit{B.S} in Computer Science (CSCI), Dean's List

\section{Experience}
\datedsubsection{\textbf{iFLYTEK Co.}, Hefei, China}{Jun 2023 - Aug 2023}
\role{Software Developer Intern}{}
  iFLYTEK is China's leading company in AI and robotics. My team is responsible for designing robots with AI functionalities like speech recognition and dialogues for children.
\begin{itemize}
  \item Built \textbf{storage crash recovery mechanism} by formatting, partitioning sd cards and data duplication in cloud.
  \item Used \textbf{ESP-IDF framework} and \textbf{FatFS} module to program in \textbf{ESP32S3 microcontroller}.
  \item Influenced products sell by millions and potentionally saved thoudsands of dollars for company.
\end{itemize}

\datedsubsection{\textbf{RecApp for USC}, Los Angeles, CA}{Jan 2022 - May 2022}
\role{Full Stack Software Developer}{}
  A student team to design an android app to manage appointments for campus recreational centers.
\begin{itemize}
  \item Worked with two other students and used \textbf{Git} for code collaboration (top 1 contributor).
  \item Wrote Java and XML in \textbf{Android Studio} for frontend design and used \textbf{Firebase} as backend.
  \item Used \textbf{Espresso} and \textbf{UIAutomator} for blackbox testing in Android UI and Google Map API.
  \item Tried by more than 20 students and received 90\% positive comments on user experience.
\end{itemize}

\section{Skills}
\begin{itemize}[parsep=0.5ex]
  \item Programming Languages: C/C++, Java, Python, PHP, Javascript, HTML/CSS, SQL, Bash/Shell, Go
  \item Tech Stack: MySQL, Firebase, Android, React.js, Docker, Django, SQLite, Node.js, MongoDB, AWS
\end{itemize}

\section{Projects}
\datedsubsection{\textbf{GoCache}}{June 2023 -- July 2023}
\role{Go, HTTP, Protobuf} {}
  A LRU \textbf{distributed cache system} written by Golang.
\begin{itemize}
  \item Supported both local cache and http-base ditributed cache by building HTTP server and client in Go.
  \item Allowed concurrent read/write from multiple \textbf{goroutine} with Go's Mutex.
  \item Optimized binary data communication between two nodes with \textbf{Protobuf}.
\end{itemize}

\datedsubsection{\textbf{Watchlists Website}}{Nov 2022 -- Dec 2022}
\role{HTML/CSS, Javascript, PHP, MySQL, AJAX, BootStrap, AWS} {}
  A website that users can create watchlists and add movies by searching though Open Movie Database api.
\begin{itemize}
  \item Use \textbf{MySQLi} in PHP for CRUD and \textbf{AJAX} for parsing API in Frontend.
  \item Self-learned \textbf{AWS} and deployed with \textbf{Apache2} server in AWS \textbf{EC2} and \textbf{RDS}
\end{itemize}

\datedsubsection{\textbf{Multithreaded Traffic Throttling Web Server}}{Sep 2022 -- Nov 2022}
\role{C++11, Socket Programming} {}
  A simple web server that can read and parse \textbf{http requests} and send corresponding http response.
\begin{itemize}
  \item Implemented with Multithreading using \textbf{C++11} so it can handle multiple connections at a time.
  \item Allowed concurrency with \textbf{lock hierarchy} to fix deadlock issues.
  \item Outputted byte streams with \textbf{traffic control} to maximize throughput stability.
\end{itemize}



\end{document}
